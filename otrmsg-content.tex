\title{%
  On the Security of Off-The-Record Messaging
  and Deniable Encryption
}
\author{%
  Daniel Bosk
}
\institute{%
  School of Computer Science and Communication\\
  KTH Royal Institute of Technology, SE-100\,44 Stockholm, Sweden\\
  \email{dbosk@kth.se}
}

\maketitle
\begin{abstract}
  The \ac{otr} messaging protocol gives an adversary information about the key.
  In this paper we describe why this is and we (hopefully) present a solution 
  to the problem.
\end{abstract}


\acresetall
\section{Introduction}

The original \ac{otr} messaging protocol was first described in 2004 by 
\citeauthor{otr2004} \cite{otr2004}.
Later improvements occured in 2007 \cite{otr2007} and 2009 \cite{multiotr2009}.
The main application at the time was \ac{im} through tools like Pidgin 
\cite{pidgin} (formerly Gaim).
\ac{im} is still used, however, the context has changed.
Since 2004 the means of communication has steered over to Facebook, Twitter, 
and other \acp{osn}, as well as messaging using smartphones instead of 
computers.
In 2010, the \ac{otr} messaging protocol was again updated and implemented for
use in the smartphone text-messaging application TextSecure \cite{textsecure}.

The desirable properties that the protocol was designed to provide are
\begin{itemize}
  \item perfect forward-secrecy for the encryption,
  \item repudiation for the authentication mechanism,
  \item plausible deniability and forgeability for the contents.
\end{itemize}
In 2013, the capabilities of the NSA and GCHQ, among others, back in 2008 were 
revealed, see e.g.~\cite{nsa1} and \cite{nsa2}.
In light of this, We claim a new adversary model is needed.
Under this adversary model, some of the properties that the \ac{otr} protocol 
was designed to provide are \emph{possibly} no longer provided, or are at least 
reduced.

We begin by describing the development of the protcols and then identify 
problems and suggest solutions to these problems.
We then present the suggested solutions again together in a new protocol.


\section{The \acs{otr} Messaging Protocol}

In this section, we will first summarize the original version of the \ac{otr} 
messaging protocol published in \cite{otr2004}.
This will be followed by the improvements after the attack published in 
\cite{di2005secure} and finally the TextSecure version of the protocol 
\cite{frosch2014secure}.

\subsection{\acs{otr} version 1 and 2 (and 3)}

As described above, the purpose of the \ac{otr} protocol was to give an online 
conversation the properties of a face-to-face real-life conversation.
The properties, and how they are achieved, are the following:
\begin{description}
  \item[Perfect forward secrecy] is achieved by continuously changing keys, for 
    every message, using the Diffie-Hellman key agreement protocol \cite{dh}.
    It is of crucial importance for Alice and Bob to forget their keys as soon 
    as possible to achieve this.

  \item[Authentication] will be achieved by using non-repudiable signatures so 
    that Bob can verify with whom he is exchanging messages.
    These keys are long-lived to make it easier for Alice and Bob to trust the 
    signatures.
    However, no message content will be signed, only the first Diffie-Hellman 
    exponent will be signed.

  \item[Integrity and repudiation] for message content will be achieved by 
    using \acp{mac}.
    Since \acp{mac} does not provide non-repudiation, Alice can later repudiate 
    sending a certain message as Bob has the means to forge it.
	The \ac{mac} employed is HMAC \cite{hmac} using a SHA-algorithm \cite{shs}.
    (SHA-1 was used originally, SHA256 subsequently.)

  \item[Forgeability and plausible deniability] of message content is achieved 
    by using malleable encryption \cite{nonmalleable}.
    This means that an adversary can do meaningful changes in the plaintext by 
    changing the ciphertext without knowing the key.
    In the case of \ac{otr}, a stream cipher is employed, namely AES \cite{aes} 
    in counter mode \cite{blockmodes}.
\end{description}

The protocol is strictly two-party, so Alice and Bob can talk privately, but no 
more people.
The original protocol \cite{otr2004} is as follows.

\begin{protocol}[\acs{otr} version 1 initialization]\label{proto:otr1init}
  Alice and Bob agree on a multiplicative group \(G\) and a generator \(g\in 
  G\).
  Alice has a signature and verification key-pair \((S_A, V_A)\) and Bob has 
  correspondingly \((S_B, V_B)\).

  Alice chooses a random \(x_1\in \Z_{\ord G}\),
  Bob chooses a random \(y_1\in \Z_{\ord G}\),
  and then they exchange the these:
  \begin{align*}
    A\to B&\colon \sign_{S_A}( g^{x_1} ), V_A \\
    B\to A&\colon \sign_{S_B}( g^{y_1} ), V_B,
  \end{align*}
  which concludes the initialization.
  \qed
\end{protocol}

Alice and Bob now has both \(g^{x_1}\in G\) and \(g^{y_1}\in G\) and can hence 
compute \(g^{x_1 y_1}\in G\) which can be used as a key.
Further, since the values are signed they can somewhat believe they are talking 
to the right person.
It is only \enquote{somewhat} since we will see an identity-misbinding attack 
against this protocol later.

After initializing the session, Alice and Bob can start their conversation 
exchanging messages \(M_1, M_2, \ldots\).
This is done in the following way.

\begin{protocol}[\acs{otr} version 1 communication]\label{proto:otr1comm}
  Assume Alice has sent \(g^{x_i}\) to Bob and she has received \(g^{y_j}\) 
  from Bob.
  Futher, let \(H\) be a collision-resistant hash function.

  When Alice wants to send a message \(m_k\) to Bob, she chooses a random 
  \(x_{i+1}\in\Z_{\ord G}\) and sends
  \begin{equation*}
    A\to B\colon g^{x_{i+1}},
      c_k = \enc_{g^{x_i y_j}}( k )\oplus m_k,
      \mac_{H( g^{x_i y_j} )}( g^{x_{i+1}}, c_1 )
  \end{equation*}
  to Bob.

  Later, when Bob wants to reply a messasge \(m_{k+1}\), he chooses a random 
  \(y_{j+1}\in\Z_{\ord G}\) and the latest \(g^{x_{i+n}}\) received from Alice, 
  and then sends
  \begin{equation*}
    B\to A\colon g^{y_{j+1}},
    c_{k+1} = \enc_{g^{x_{i+n} y_j}}( k+1 )\oplus m_{k+1},
    \mac_{H( g^{x_{i+n} y_j} )}( g^{y_{j+1}}, c_{k+1} )
  \end{equation*}
  to Alice.

  Once Alice knows that a message sent to Bob has been received, she publishes 
  the \ac{mac} key \( H( g^{x_i y_j} ) \) for that message as part of her next 
  message.
  \qed
\end{protocol}

We can see in Protocol \ref{proto:otr1comm} that we get a continuous 
Diffie-Hellman exchange by using the transmission of each message for 
exchanging the value to be used for the next message.
Hence we never use the same key twice.
We remain authenticated as well, since we provide authenticity to the next 
Diffie-Hellman value using a keyed \ac{mac} with a previously authenticated 
key.
However, since the authentication is done using \acp{mac}, Bob can never prove 
to anyone else what Alice has written.
And after Alice publishes the \ac{mac} key, anyone can create messages which 
authenticates using that \ac{mac} key.
Thus, no one can use the \ac{mac} to verify the message in the future.

For the perfect forward secrecy, we need Alice and Bob to forget the keys 
\(g^{x_i y_j}\) as soon as possible.
We observe that Alice needs to keep her secret values \(x_i, x_{i+1}, \ldots, 
x_{i+n}\) until she has received a message from Bob using a newer value.
If she does not, then she cannot decrypt any message from Bob using that value 
in the key.

\citeauthor{di2005secure} \cite{di2005secure} found an attack on the 
initialization of this protocol (Protocol \ref{proto:otr1init}).
The attack results in an identity misbinding, described in general in 
\cite{ake}.
This means that Eve can make Alice believe she is talking to Bob while Bob 
believes he is talking to Eve and Alice.
Alice and Bob can compute the secret key, but not Eve, she can only get Bob to 
believe he is talking to someone else.
This is accomplished in the following manner:
\begin{align*}
  A\to E &\colon g^x, \sign_{S_A}( g^x ), V_A \\
  E\to B &\colon g^x, \sign_{S_E}( g^x ), V_E \\
  B\to E &\colon g^y, \sign_{S_B}( g^y ), V_B \\
  E\to A &\colon g^y, \sign_{S_B}( g^y ), V_B \\
\end{align*}
The following solution is presented in \cite{di2005secure}:
\begin{align*}
  A\to B &\colon g^{x} \\
  B\to A &\colon g^{y} \\
  A\to B &\colon A,
    \sign_{S_A}( g^{y}, g^{x} ),
    \mac_{H( g^{x y} )}( 0, A ),
    V_A \\
  B\to A &\colon B,
    \sign_{S_B}( g^{x}, g^{y} ),
    \mac_{H( g^{x y} )}( 1, B ),
    V_B \\
\end{align*}
This way, Alice and Bob authenticates over the secret \(g^{xy}\) which Eve does 
not know.

This attack led to the second version of the protocol, where the authentication 
step was improved.
Another improvement added to this protocol was to hide the public keys of the 
participants from passive adversaries.
The resulting authentication protocol follows as described in \cite{otr2007}.

\begin{protocol}[Simplified \acs{otr} version 2 and 3 initialization]
  \label{proto:otr2init}
  As in Protocol \ref{proto:otr1init}, Alice and Bob agree on a multiplicative 
  group \(G\) and a generator \(g\in G\).
  Alice has a signature and verification key-pair \((S_A, V_A)\) and Bob has 
  correspondingly \((S_B, V_B)\).

  Alice chooses a random key \(r\in \K\), from the keyspace of the encryption 
  algorithm \(\enc\), and \(x_1\in \Z_{\ord G}\).
  Bob chooses a random \(y_1\in \Z_{\ord G}\).
  Then they exchange the following:
  \begin{align*}
    A\to B &\colon \enc_r( g^{x_1} ), H( g^{x_1} ) \\
    B\to A &\colon g^{y_1} \\
    A\to B &\colon r
  \end{align*}

  Next, Alice and Bob derives \ac{mac} keys \(a_1, a_2, b_1, b_2\) and 
  encryption keys \(a_3, b_3\) from \(g^{x_1 y_1}\).
  Alice computes \[M_A = \mac_{a_1}( g^{x_1}, g^{y_1}, V_A )\textand X_A 
  = ( V_A, \sign_{S_A}( M_A ) ).\]
  She then sends
  \begin{align*}
    A\to B\colon c = \enc_{a_3}( X_A ), \mac_{a_2}( c )
  \end{align*}
  to Bob.
  Bob verifies the received values and computes \[M_B = \mac_{b_1}( g^{y_1}, 
  g^{x_1}, V_B )\textand X_B = ( V_B, \sign_{S_B}( M_B ) ).\]
  He then sends
  \begin{align*}
    B\to A\colon c^\prime = \enc_{b_3}( X_B ), \mac_{b_2}( c^\prime )
  \end{align*}
  to Alice, who verifies the received values which concludes the 
  initialization.
  \qed
\end{protocol}

\citeauthor{otr2007} \cite{otr2007} adds another authentication protocol to 
this.
Protocols \ref{proto:otr1init} and \ref{proto:otr2init} both requires Alice and 
Bob to be able to verify each other's public key.
To increase usability they add an adaption of the \ac{smp} \cite{smp} which 
allows them to use a shared low-entropy secret to identify each other instead 
of the fingerprints of their public keys.

\subsection{\acs{otr} with Improved Authentication}

\dots described in \cite{otr2007}.

\begin{protocol}[The Socialist Millionaires' Protocol]
  \dots
  \qed
\end{protocol}

\subsection{TextSecure}

The TextSecure protocol was first described and analysed by 
\citeauthor{frosch2014secure} in \cite{frosch2014secure}.
The TextSecure protocol builds on the \ac{otr} protocol, but it has made some 
changes to adapt it to the setting of smartphone messaging.

\dots

\begin{protocol}[TextSecure pre-initialization]
  Each user generate prekeys \(g^{x_1}, \ldots, g^{x_n}\).
  These are then signed and uploaded to a server.
  \qed
\end{protocol}

\dots

\begin{protocol}[TextSecure initialization]
  \dots
  \qed
\end{protocol}

\dots

\begin{protocol}[TextSecure communication]
  \dots
  \qed
\end{protocol}

\dots


\section{Multi-Party \acs{OTR}}

\dots described in \cite{multiotr2009}.

\begin{protocol}[Multi-Party \acs{otr}]
  \dots
  \qed
\end{protocol}


\section{The Main Ideas}

The protocol does not consider the intelligence agency of a nation state as 
a possible adversary, which should be considered from what we know today.
In this new adversarial setting, where Eve controls the network and can 
possibly compromise a computer system at will, some of the properties in the 
\ac{otr} protocol might no longer be valid.
More importantly, the nation state does not have to present non-repudiable 
evidence to a third-party judge, it just has to convince itself.
We argue that in this setting the algorithms employed in this protocol are 
\enquote{too secure} for our own good.
We will elaborate on this.

The arguments in \cite{otr2004} for forgeability using malleable encryption and 
publishing the \ac{mac} keys only hold if the adversary cannot trust the source 
of the transcript.
Using the adversary just described, she can ultimately trust the transcript 
since she collected it herself from the network.
She then knows exactly which ciphertext Alice sent to Bob, and which ciphertext 
Bob replied.
She knows the lengths of these ciphertexts and she knows the time they were 
sent and the order in which they arrived.
She knows the \acp{mac} for each ciphertext as well, and that Alice and Bob 
sent them like this.
She knows that Alice is the author of a message because she observes when Alice 
publishes the \ac{mac} key.
When Eve compromises Bob's system, she might be able to see some history of the 
conversation in his client.
She can then infer who wrote what thanks to the metadata in her records, e.g.\ 
message length and ordering.

Furthermore, Eve also gets some information about the key from the ciphertext 
and \ac{mac}.
Eve can use the \ac{mac} to discard false keys for the cipher as in Algorithm 
\ref{alg:falsekeys}.
Hence, by having the \ac{mac} key depend on the encryption key, we 
automatically decrease the number of spurious keys and thus also reduce our 
possibility for deniability.
This can be seen in the following equations:
\begin{equation*}
  \enc_k( m ) = c = \enc_{k^\prime}( m^\prime )
\end{equation*}
and if \(k\neq k^\prime\) then
\begin{equation*}
  \Pr\left[
    \mac_{H(k)}( c ) = \mac_{H(k^\prime)}( c )
  \right] < 1.
\end{equation*}

\begin{algorithm}
  \caption{%
    Algorithm for finding possible plaintexts and discarding false keys.
  }
  \label{alg:falsekeys}
  \begin{algorithmic}
    \Function{IsTrueKey}{$C, M, k$}\Comment{%
      Ciphertext $C$, \ac{mac} $M$, key $k$.
    }
      \If{$\mac_{H(k)}( C ) = M$}
        \State \Return True
      \EndIf
      \State \Return False
    \EndFunction
    \Statex
    \Function{Decrypt}{$C, M$}\Comment{Ciphertext $C$ and \ac{mac} $M$.}
      \State Let $P = \emptyset$.
      \ForAll{$k\in \K$}\Comment{Keyspace $\K$}
        \If{\Call{IsTrueKey}{$C, M, k$}}
          \State Append $\dec_k( C )$ to $P$.
        \EndIf
      \EndFor
      \State \Return $P$
    \EndFunction
  \end{algorithmic}
\end{algorithm}

The be able to get deniability in the given scenario, Alice and Bob need to be 
able to modify the plaintext without modifying the ciphertext.
This can be accomplished by using the \ac{otp}, since then you can easily 
compute the key from a given plaintext-ciphertext pair.

They also need a \ac{mac} key independent from the encryption key.
Then they can change the encryption key and the plaintext, but the ciphertext 
and \ac{mac} remains the same.
Instead of using \(g^{xy}\in G\) and encryption key and \(H(g^{xy})\) as 
\ac{mac} key, Alice and Bob can increase the size \(\ord G\) of the group \(G\) 
and use the \(n\) least significant bits of \(g^{xy}\) for \ac{mac} and the 
rest for encryption.
Choosing a group which gieves a keysize of 2048 bits, using 256 bits for 
\ac{mac} leaves 1792 bits for encryption with \ac{otp}.
That is 224 bytes, which is larger than what Twitter allows for a tweet (140 
characters).
This way, when using the \ac{otp} they can keep the \ac{mac} key fixed while 
adapting the encryption key to their new plaintext, then hand the complete key 
(both \ac{mac} and encryption) to the adversary.

The adversary Eve has two options for verifying whether a given key \(k\) is 
true or false.
First, she can find the discrete logarithm \(s = \log_g( k )\) of \(k\) in the 
base \(g\in G\).
Then she has to factor \(s = xy\) into \(x\) and \(y\) which can be verified 
using the recorded values \(g^x\) and \(g^y\).
Second, she can find the discrete logarithms of the recorded \(g^x\) and 
\(g^y\) to verify whether \(k = g^{xy}\) it true or not.

However, it can be argued that this is the case already with the current 
protocol.
The only difference is that Alice and Bob cannot give Eve any key to 
\enquote{prove their innocense}, they can just deny because they do not 
remember the keys (perfect forward secrecy).

An improvement would be if we could also lie to Eve about \(g^x\) and \(g^y\) 
to convince her about \(g^{xy}\).
This might be possible if we can introduce some randomness in the transmission 
of \(g^x\) and \(g^y\).


\section{A More Secure \acs{otr} Messaging Protocol}

One way of getting more possibilities for lying would be to introduce some 
randomness in the Diffie-Hellman key exchange.
This randomness \(r\) must be easy to change in relation to the chosen exponent 
\(x\), so that we afterwards can adjust the \(r\) to the change from \(x\) to 
\(x^\prime\) in \(g^{x^\prime}\).


\section{Using \acs{otr} Messaging in \acsp{osn}}

In \acp{osn} like Facebook, people can still communicate through \ac{im}, but 
this is not the main form of communication.
Twitter is a good example of the new form of communication, where users 
communicate by positing up-to-140-characters long messages to each other's 
feeds.
This means more asynchronous communication.
And the same applies for smartphones, with the exception that Twitter feeds are 
always public.
This was the motivation for TextSecure \cite{TSasynch}.

\dots


%\begin{filecontents}{otrmsg.submission.bbl}
%\end{filecontents}
\printbibliography
